\section{Introduction}\label{sec:introduction}
% motivation

A relational database management system (DBMS) can execute a given query using
many alternative ways with widely varying evaluation costs. 
The process of identifying the most efficient plan for executing a query among
these alternatives is termed as query optimization. 
The query optimizer contains a collection of rules for rewriting relational
expressions in a query into alternative, equivalent expressions that can be more
efficiently computed. Two expressions in relational-algebra are equivalent 
if they generate the same set of tuples on every possible database instance,
that satisfies all the integrity constraints specified in the database schema.
 
It is challenging for DBMS developers to ensuring the validity of query
rewriting rules. Since the general problem of determining the equivalence of two
expressions is an undecidable problem, developers tend to construct rules
in an ad-hoc manner. This approach results in incorrect rewrites that introduce
critical, subtle correctness bugs in DBMSs\todo{~\cite{}}.

% close related work
In general, deciding whether two given SQL queries are semantically equivalent
is an undecidable problem(\QZ{cite}).
There has been substantial previous work done in this filed.
% introduce not so related work
Some of the work has been focussed on how to identify a subset of relational
algebra that enables proving that semantical equivalence is decidable under set
definition \QZ{cite} or under bag definition \QZ{cite}.
This type of work focusses on the theoretical analysis of an approach to this
problem, which hardly leads to a system that can be applied to real world sql
queries.

% UW works
Another previous approach is converting SQL queries into a simple algebraic
structure, then using a decision procedure to find the isomorphisms and
homomorphisms between algebraic expressions to proving equivalence under set or
bag definition.
There have been two systems implemented based on different algebraic structure
by this approach.
One is \textbf{COSETTE} \QZ{cite} that uses \textbf{k-relations} as the
algebraic structure to interpret SQL queries.
The other is \textbf{UDP} \QZ{cite} that uses \textbf{unbounded semiring} to
interpret SQL queries.
% why it is bad
While these systems have been able to prove equivalence among some pair of sql
queries under set or bag definition, it has one significant shortcoming.
These approaches do not model the semantics of many widely used features in SQL
queries, such as arithmetic operations, NULL values, and filter predicates,
which limits the usage of such approach in real world queries.

% contribution
The first contribution of this paper is proposing to use SMT solver, which is a
widely used technique in programming verification community, to decide the
semantically equivalence of sql queries under \textbf{set} definition.
This approach views each SQL query as a loop-free and guaranteed terminated
program.
This approach allows us to use logic constrains to encode the semantic of large
subset of SQL features, including filter operation with all common used
predicate, extended projection with complex arithmetic operations, three
different type of join operation, union operation, aggregate operation and NULL
values.
Because this problem is undecidable, this approach uses over-approximation to be
sound but incomplete.

The second contribution is an implemented tool \sys that can automatically
decide the equivalence of sql queries under set definition.
Because deciding the semantical equivalence of a given pair of SQL queries is an
undecidable problem, \sys is sound but incomplete.
Sound implies if \sys gives an equivalent decision, the given pair of SQL
queries are semantically equivalent.
By Incomplete we mean that if \sys gives an inequivalent decision, the given
pair of SQL queries might be indeed semantically equivalent. In other words, our
system \sys may result in some false negatives.
We apply \sys to an open benchmark set from calcite\QZ{cite}.
The evaluation result shows that \sys can prove semantical equivalence for more
pairs of sql queries than \textbf{UDP} under set definition.
We also apply \sys to a large query dataset.
The result shows that there are over 10 percent real word queries currently in
use at a reputed company xxxxx that contains at least one subquery is
semantically equivalent or be contained by other queries' subquery, which
indicates a big opportunity for optimization.

% Paper outline.
The rest of this paper is organized as follows.
\autoref{sec:formalize} formally defines the target SQL query and the problem.
\autoref{sec:overview} use a simple example to demonstrate the whole system.
and \autoref{sec:approach} presents the \sys in details.
\autoref{sec:evaluation} presents an empirical evaluation of \sys.
\autoref{sec:related-work} compares our contributions to related work, and
\autoref{sec:conclusion} concludes.
