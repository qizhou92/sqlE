\documentclass{vldb}

\newcommand{\paperTitle}{Proving Equivalence of SQL Queries Using\\
Satisfiability Modulo Theories} 
\newcommand{\paperKeywords}{Query Rewriting, Query Optimization, Satisfiability
Modulo Theories}
\newcommand{\paperAuthors}{}

% Define to fix sigplanconf.
% \doi{}

\setlength{\paperheight}{11in}
\setlength{\paperwidth}{8.5in}

\usepackage[square,comma,numbers,sort&compress]{natbib}
%\usepackage[hyphens]{url}
\usepackage{hyperref}

% hyperref itself
\hypersetup{
 pdfauthor = {\paperAuthors}, 
 pdftitle = {\paperTitle}, 
 pdfkeywords = {\paperKeywords}, 
 pdfborder={ 0 0 0 }
}

\usepackage[usenames,dvipsnames]{xcolor}
\usepackage{amsmath,amsopn,amssymb}
\usepackage{subfig}
\usepackage{endnotes,microtype,xspace,graphicx,fancyvrb,multirow}
\usepackage{booktabs}
\usepackage{array,underscore,relsize}
\usepackage[T1]{fontenc}
\usepackage{times}
\usepackage{fancyhdr,lastpage}
\usepackage{enumitem}
% \usepackage[labelfont=bf,font=small,skip=5pt]{caption}

%\usepackage{lmodern}

% aliascnt: counter stuff that works with theorem environments.
\usepackage{aliascnt}

% algorithm2e: algorithms
\usepackage[linesnumbered]{algorithm2e}

% semantic: inference rules
\usepackage{semantic}

% stmaryrd: more math fonts (namely mathbb)
\usepackage{stmaryrd}

% floatrow: putting two figures side by side
\usepackage{floatrow}

\usepackage{mathtools}

\pagestyle{fancy}
\fancyhf{}
\renewcommand{\headrulewidth}{0pt}
\cfoot{\thepage}

\newcommand{\sys}{\mbox{\textsc{Shara}}\xspace}

% cmds: typesetting commands
\renewcommand{\ttdefault}{pxtt}

\newcommand{\URL}{\url}
\newcommand{\cc}[1]{\mbox{\smaller[0.5]\texttt{#1}}}

%\clubpenalty=10000
%\widowpenalty=10000

%\linespread{1.2}

\fvset{fontsize=\scriptsize,xleftmargin=8pt,numbers=left,numbersep=5pt}

\input{code/fmt}
\newcommand{\figrule}{\hrule width \hsize height .33pt}
\newcommand{\coderule}{\vspace{-0.4em}\figrule}

\setlength{\abovedisplayskip}{0pt}
\setlength{\abovedisplayshortskip}{0pt}
\setlength{\belowdisplayskip}{0pt}
\setlength{\belowdisplayshortskip}{0pt}
\setlength{\jot}{0pt}

\def\Snospace~{\S{}}
%\renewcommand*\sectionautorefname{\Snospace}
\def\sectionautorefname{\Snospace}
\def\subsectionautorefname{\Snospace}
\def\subsubsectionautorefname{\Snospace}
\def\chapterautorefname{\Snospace}
\def\equationautorefname{Eqn.}

%\renewcommand{\figurename}{Fig.}
%\def\figureautorefname{\figurename}
\newcommand{\subfigureautorefname}{\figureautorefname}

%\numberwithin{equation}{section}
\newcommand{\yes}{Y}
\newcommand{\no}{}

% sema
\newcommand{\shl}{\ \cc{<}\cc{<}\ }
\newcommand{\shr}{\ \cc{>}\cc{>}\ }

\if 0
\renewcommand{\topfraction}{0.9}
\renewcommand{\dbltopfraction}{0.9}
\renewcommand{\bottomfraction}{0.8}
\renewcommand{\textfraction}{0.05}
\renewcommand{\floatpagefraction}{0.9}
\renewcommand{\dblfloatpagefraction}{0.9}
\setcounter{topnumber}{10}
\setcounter{bottomnumber}{10}
\setcounter{totalnumber}{10}
\setcounter{dbltopnumber}{10}
\fi

\newif\ifdraft\drafttrue
\newif\ifnotes\notestrue
\ifdraft\else\notesfalse\fi

% per-author notes:
% ref. http://en.wikibooks.org/wiki/LaTeX/Colors
\newcommand{\BH}[1]{\textcolor{Green}{BH: #1}}
\newcommand{\CV}[1]{\textcolor{Orange}{BH: #1}}
\newcommand{\QZ}[1]{\textcolor{Red}{QZ: #1}}
\newcommand{\AJ}[1]{\textcolor{green}{AJ: #1}}

% \newcommand{\TODO}[1]{\textcolor{red}{TODO: #1}}

%% Ensure ligatures (e.g., ``fine official flag'') can be copy/pasted from PDF.
\input{glyphtounicode}
\pdfgentounicode=1

\newcolumntype{R}[1]{>{\raggedleft\let\newline\\\arraybackslash\hspace{0pt}}p{#1}}

% include macros
\newcommand{\includepdf}[1]{
  \includegraphics[width=\columnwidth]{#1}
}

\newcommand{\includeplot}[1]{
  \resizebox{\columnwidth}{!}{\input{#1}}
}

% list
\newcommand{\squishlist}{
\begin{itemize}[noitemsep,nolistsep]
  \setlength{\itemsep}{-0pt}
}
\newcommand{\squishend}{
  \end{itemize}
}

%% NOTE.
%%  to use circled number in caption, use
%%   (e.g., \protect\C{1})
%%
\usepackage{tikz}
\newcommand*\C[1]{%
\begin{tikzpicture}[baseline=(C.base)]
\node[draw,circle,inner sep=0.2pt](C) {#1};
\end{tikzpicture}}

\newcommand*\BC[1]{%
\begin{tikzpicture}[baseline=(C.base)]
\node[draw,circle,fill=black,inner sep=0.2pt](C) {\textcolor{white}{#1}};
\end{tikzpicture}}

\newcommand{\PP}[1]{
\vspace{2px}
\noindent{\bf #1}
}


% std: standard math commands and environments.
\input{std}

% shorthand: paper-specific shorthand commands.
\input{shorthand}
\input{sqlcmd}
% rev: revision information: generated by build system
\input{rev}
\vldbTitle{\paperTitle}
\vldbAuthors{\paperAuthors}
\vldbDOI{https://doi.org/10.14778/xxxxxxx.xxxxxxx}
\vldbVolume{xx}
\vldbNumber{yyy}
\vldbYear{2019}

\title{\paperTitle}
\author{\paperAuthors}

\begin{document}

\maketitle

\begin{abstract}

The design of the buffer manager in database management systems (DBMSs)
is influenced by the performance characteristics of volatile
memory (DRAM) and non-volatile storage (e.g., SSD).
The key design assumptions have been that the data must be migrated to DRAM 
for the DBMS to operate on it and that storage is orders of magnitude slower
than DRAM. But the arrival of new non-volatile memory (NVM) technologies that
are nearly as fast as DRAM invalidates these previous assumptions.

This paper presents techniques for managing and designing a multi-tier storage
hierarchy comprising of DRAM, NVM, and SSD.
Our main technical contributions are a multi-tier buffer manager and a storage
system designer that leverage the characteristics of NVM.
We propose a set of optimizations for maximizing the utility of data migration
between different devices in the storage hierarchy. 
We demonstrate that these optimizations have to be tailored based on 
device and workload characteristics. 
Given this, we present a technique for adapting these optimizations to achieve a
near-optimal buffer management policy for an arbitrary workload and storage
hierarchy without requiring any manual tuning.
We finally present a recommendation system for designing a multi-tier storage
hierarchy for a target workload and system cost budget.
Our results show that the NVM-aware buffer manager and storage system designer
improve throughput and reduce system cost across different transaction and
analytical processing workloads.
\end{abstract}

% abstract.tex: DEP, contains abstract
\section{Introduction}\label{sec:introduction}
% motivation

A relational database management system (DBMS) can execute a given query using
many alternative ways with widely varying evaluation costs. 
The process of identifying the most efficient plan for executing a query among
these alternatives is termed as query optimization. 
The query optimizer contains a collection of rules for rewriting relational
expressions in a query into alternative, equivalent expressions that can be more
efficiently computed. Two expressions in relational-algebra are equivalent 
if they generate the same set of tuples on every possible database instance,
that satisfies all the integrity constraints specified in the database schema.
 
It is challenging for DBMS developers to ensuring the validity of query
rewriting rules. Since the general problem of determining the equivalence of two
expressions is an undecidable problem, developers tend to construct rules
in an ad-hoc manner. This approach results in incorrect rewrites that introduce
critical, subtle correctness bugs in DBMSs\QZ{cite}.

% close related work
In general, deciding whether two given SQL queries are semantically equivalent
is an undecidable problem(\QZ{cite}).
There has been substantial previous work done in this filed.
% introduce not so related work
Some of the work has been focussed on how to identify a subset of relational
algebra that enables proving that semantical equivalence is decidable under set
definition \QZ{cite} or under bag definition \QZ{cite}.
This type of work focusses on the theoretical analysis of an approach to this
problem, which hardly leads to a system that can be applied to real world sql
queries.

% UW works
Another previous approach is converting SQL queries into a simple algebraic
structure, then using a decision procedure to find the isomorphisms and
homomorphisms between algebraic expressions to proving equivalence under set or
bag definition.
There have been two systems implemented based on different algebraic structure
by this approach.
One is \textbf{COSETTE} \QZ{cite} that uses \textbf{k-relations} as the
algebraic structure to interpret SQL queries.
The other is \textbf{UDP} \QZ{cite} that uses \textbf{unbounded semiring} to
interpret SQL queries.
% why it is bad
While these systems have been able to prove equivalence among some pair of sql
queries under set and bag definition, it has one significant shortcoming.
These approaches do not model the semantics of many widely used features in SQL
queries, such as arithmetic operations, NULL values, and complicated filter predicates,
which limits the usage of such approach in real world queries.

% contribution
The first contribution of this paper is proposing to use SMT solver, which is a
widely used technique in programming verification community, to decide the
semantically equivalence of sql queries under \textbf{set} definition.
This approach views each SQL query as a loop-free and guaranteed terminated
program.
This approach allows us to use logic constrains to encode the semantic of large
subset of SQL features, including filter operation with all common used
predicate, extended projection with complex arithmetic operations, three
different type of join operation, union operation, aggregate operation and NULL
values.
Because this problem is undecidable, this approach uses over-approximation to be
sound but incomplete.

The second contribution is an implemented tool \sys that can automatically
decide the equivalence of sql queries under set definition.
Because deciding the semantical equivalence of a given pair of SQL queries is an
undecidable problem, \sys is sound but incomplete.
Sound implies if \sys gives an equivalent decision, the given pair of SQL
queries are semantically equivalent.
By Incomplete we mean that if \sys gives an inequivalent decision, the given
pair of SQL queries might be indeed semantically equivalent. In other words, our
system \sys may result in some false negatives.
We apply \sys to an open benchmark set from calcite\QZ{cite}.
The evaluation result shows that \sys can prove semantical equivalence for more
pairs of sql queries than \textbf{UDP} under set definition.
We also apply \sys to a large query dataset.
The result shows that there are over 10 percent real word queries currently in
use at a reputed company xxxxx that contains at least one subquery is
semantically equivalent or be contained by other queries' subquery, which
indicates a big opportunity for optimization.

% Paper outline.
The rest of this paper is organized as follows.
\autoref{sec:formalize} formally defines the target SQL query and the problem.
\autoref{sec:overview} use a simple example to demonstrate the whole system.
and \autoref{sec:approach} presents the \sys in details.
\autoref{sec:evaluation} presents an empirical evaluation of \sys.
\autoref{sec:related-work} compares our contributions to related work, and
\autoref{sec:conclusion} concludes.


\AJ{Need to break down approach into three sections similar to UDP paper. Example:
\begin{enumerate}[noitemsep,nolistsep,label=\protect\BC{\arabic*},leftmargin=7ex]
\setlength{\itemsep}{-0pt}
\item Axiomatic Foundations (?)
\item Aggregates etc.
\item Decision Procedures (?)
\end{enumerate}
}

\section{Overview}\label{sec:overview}
%
In this section, we demonstrate our approach by two examples.
%
We first introduce two pairs of semantically equivalent SQL queries and the table schema from the open database
framework calcite\QZ{cite}.
%
Then we show the fundamental reasons that the previous approach cannot prove their equivalence.
%
Finally, we demonstrate how our approach applies Satisfiability Modulo Theory (SMT) solver to prove the equivalence 
of these two examples. 
%
\begin{figure}
\input{code/q1.sql}
\caption{%
    A pair of sql queries from calcite.
  }\label{fig:q1}
\end{figure}
%

%introduce the first query
\autoref{fig:q1} contains a pair of SQL queries and its table schema.
%
$EMP$ is a table that has three columns $EMPNO$, $ENAME$ and $DEPTNO$.
%
$EMPNO$ and $DEPTNO$ are integer type, and $ENAME$ is string type.
%
The query $q_0$ first selects all tuples from table $EMP$ that $DEPTNO$ is equal to
10 and calls the result as $t$.
%
Then $q_0$ selects all tuples from $t$ such that $DEPTNO$ plus five is larger than $EMPNO$.
%
The query $q_1$ first selects all tuples from table $EMP$ that $DEPTNO$ is equal to
ten and calls the result as $t1$.
%
Then $q_1$ selects all tuples from $t1$ such that $EMPNO$ is less than 15.
%
These two sql queries are obviously semantic equivalent 
because once the $DEPTNO$ is equal to ten, $DEPTNO$ plus five equals to 15.
%


%introduce the second query
%
\begin{figure}
\input{code/q2.sql}
\caption{%
    A pair of sql queries from calcite.
  }\label{fig:q2}
\end{figure}
%
\autoref{fig:q2} contains a pair of SQL queries with the same table schema as above.
%
The query $q_0$ selects all tuples from table $EMP$ such that $EMPNO$ equals to 10, and
$EMPNO$ is not NULL.
%
The query $q_1$ selects all tuples from table $EMP$ such that $EMPNO$ equals to 10.
%
These two sql queries are semantically equivalent. 
%
Because based on three value logic, if $EMPNO$ equals to 10 be evaluated as true, then 
$EMPNO$ is not NULL.  

%talk about the previous approach
Previous approaches such as Cosette\QZ{cite} and Udp\QZ{cite} fails to prove the semantic
equivalence of these two pairs of queries.
%
Cosette and Udp use K-relations and U-semiring respectively as a simple algebra structure for
representing sql queries.
%
After sql queries being translated into nominalized form in a simple algebra strcutre, then Cosette and
Udp applies a set of rewrite rules on algebra expression.
%
If an isomorphism can be found between two rewrited algebra expressions, then the two sql queries are equivalent.

%Why it fails for the first example
For proving the equivalence of the pair of queries in \autoref{fig:q1}, two queries can be 
represented as the following algebra representation.
%
\[q1 : \lbrack t.DEPTNO = 10 \rbrack \times \lbrack t.DEPTNO + 5 > t.EMPNO \rbrack \times EMP(t)\]
%
\[q2:  \lbrack t.DEPTNO = 10 \rbrack \times \lbrack 15 > t.EMPNO \rbrack \times EMP(t)] \]
%
The previous approaches cannot prove these two queries are equivalent.
%
Because this proof requires modeling the semantic of arithmetic operations and combining the semantic of
each predicate.
%
The automatic proof system needs to be able to infer the fact that the predicate 
$\lbrack t.DEPTNO + 5 > t.EMPNO \rbrack$ and predicate $\lbrack 15 > t.EMPNO$ are identical if the predicate
$\lbrack t.DEPTNO = 10 \rbrack$ holds.


%Why it fails for the second example
For proving the equivalence of the pair of queries in \autoref{fig:q2}, two queries can be 
represented as the following algebra representation.
%
\[q1 : \lbrack t.DEPTNO = 10 \rbrack \times \lbrack NOT\_NULL(t.DEPTNO) \rbrack \times EMP(t)\]
%
\[q2:  \lbrack t.DEPTNO = 10 \rbrack \times EMP(t)] \]
%
The previous approaches cannot prove these two queries are equivalent.
%
Because this proof requires modeling the three logic system for null value, modeling the semantic of 
equal predicates and combining the semantic of each predicate.
%
The automatic proof system needs to be able to infer that if $\lbrack t.DEPTNO = 10 \rbrack$ holds, 
then $t.DEPTNO$ cannot be an NULl value.
%
The fact needs to be inferred is non-trivial.
%
Because for some binary predicates, the column can be NULL when the predicate be evaluated as
true.
%
For example, predicate $\lbrack OR TRUE t.DEPTNO \rbrack$ holds in three logic system when $t.DEPTNO$ is NULL.

% key reasons why it fails
There are two main reasons that the approaches of translating SQL queries into simple algebra expressions 
and deciding the equivalence by applying re-write rule to find isomorphism are limited.
%
First, as the above example, each predicate has certain sematically meaning that would affect other predicates.
%
These approaches unable to modeling the semantic meaning of all predicates and combining these semantic of these
predicates.
%
Second, not only predicates, but also some operator has certain semantically would affect other predicates.
%
For example, for the left outer join, when the tuples in left table fails to find the matching tuples 
in the right tuples, the new tuples is concatenation of left tuple with null values.
%
If $NOT\_NULL$ predicates be applied to the new tuples, it eliminates all tuples that columns from 
right table are null. 
%
The left outer join actually be reduced to inner join when certain predicates applied.


%
In order to prove the semantic equivalence of queries in \autoref{fig:q1} and \autoref{fig:q2}.
%
We proposed use satisfiability modulo theories(SMT) solver, which is 
a widely accepted technique in program verification problems, to verifying the equivalence between SQL queries.
%

We prove the semantic equivalence of SQL queries under \textbf{set} definition.
%
It means two SQL queries are be considered semantic equivalent if and only if for all valid input tables, 
the output tables are the same of applying these two queries on the same input tables and eliminating all duplicates
tuples in the output tables.
%
We choose the semanticaly equivalence under the \textbf{set} definition becuase \QZ{related to motivation}.
%
We prove the equivalence between two queries by proving two containment relations that each query contains the other.
%
We define containment relations under set definition as query $q_1$ contains query $q_2$ if and only if
for all valid input tables, the tuples in the output table of applying $q_2$ on input tables all shows in
the output table of applying $q_1$ on the same input tables.
%
In other words, if there is no tuple in the output table of $q_2$ that not int output table of $q_1$ for the same
valid inputs, then $q_1$ contains $q_2$.
%
The formation definitions of containment and equivalent are given in \autoref{sec:formalize}.

%
We have reduced the problem of verifying the equivalence of sql queries to verify the containment relations of sql
queries.
%
We use pairs of queries in \autoref{fig:q1} and \autoref{fig:q2} as examples to demonstrate how to verify 
containment relations of sql queries.
%

% the intuition of the design
The key observation for queries in \autoref{fig:q1} and \autoref{fig:q2} is that each tuple in the output table is
corresponding a tuple in the original input table $EMP$.
%
The semantic of these queries is filtering out tuples in table $EMP$ that does not satisfy a set of predicates.
%
Furthermore, if we only consider sql query that contains table, filter, project and inner join operation, 
then each tuples in the output table is constructed by choosing one tuple
from each input table.
%
This type of queries can be viewed as applying a program, that takes one tuple
from each input table that returns a new tuple in the output table if certain predicates are satisfied, on the 
cartesian product of all input tables.
%
Queries that contains left outer join, full outer join, union and aggregate will be discussed in \QZ{section}.

%
In this example, $q_1$ and $q_2$ in \autoref{fig:q1} can be viewed as multiple execution of the program 
$p_0$ and $p_1$ that is shown in \autoref{fig:p1}
%
\begin{figure}
\input{code/p1.java}
\caption{%
    programs that represent queries.
  }\label{fig:p1}
\end{figure}
%
$p_1$ and $p_2$ takes a tuple from the input table $EMP$ and outputs a tuple if it satisfies certain conditions.
%
We can prove query $q_1$ contains $q_2$ by proving following statement:
For a given input which is a tuple from $EMP$, if program $p_2$ returns a tuple, 
then $p_1$ returns the same tuple.

%
Proving this statement needs two steps.
%
First we need to prove for an arbitrary input tuple $EMPNO,ENAME,DEPTNO$, if program $p_2$ 
reaches the return statement, then program $p_1$ also reaches the return statement.
%
Then we need to prove for an arbitrary input tuple that program $p_1$ and $p_2$ both reach the return statement, 
program $p_1$ and $p_0$ returns the same tuple.
%
This condition holds trivially for this example, because both sql queries directly return the input tuple.
%
But for queries that contain projection, this condition needs to be verified.

%how to do it.
The key here is that program $p_0$ and $p_1$ are loop-free and guaranteed terminated program due to the nature of
SQL queries.
%
Verifying relational properties for loop-free and guaranteed terminated programs is 
a well-studied problem in program language community.
%
Using satisfiability modulo theories(SMT) solver is a widely accepted technique for loop-free and guaranteed
terminated programs' verification problems.
%

%Explain SMT
Satisfiability modulo theories solver is a tool that decides if a given first order logic formula has an
interpretation for the variables that satisfy the formula.
%
If the formula is satisfiable, then the solver returns a model of variables that satisfy this formula.
%
For example, given the formula $x>0 /\ x < 5$, the SMT solver returns $SAT$, which means this formula is 
\textbf{satisfiable}.
%
Because $x$ can be $1$, and $1$ is greater then 0 and less than 5.
%
Given the formula $x > 10 /\ x < 5$, then SMT solver returns $UNSAT$, which means this formula is 
\textbf{unsatisfiable}.
%
Because there is no value for $x$ that can satisfy this formula.
%
For verifying the program properties, programs and properties are encoded as logic constrains, and using SMT
solver to check the satisfiability of the logic constrains to decides if the properties holds for the programs.

%
In this example, because the input is a vector of three elements, we create three variables $SEMPNO$,$SENAME$ and 
$SDEPTNO$ in the logic constrains that each 
corresponding an element in the input vector.
%
The condition of program $p_0$ reaches the return statement can be encoded into an first order logic constrains \\
$c_0 := SDEPTNO = 10 \land SDEPTNO + 5 > SEMPNO$.
%
And the logic constrains for program $p_1$ reaches the return statement is $c_1 := SDEPTNO = 10 \land 15 > SEMPNO$.
%
For verifying $p_1$ reaches the return statement implies $p_0$ reaches the return statement, 
we need to verify for any input that satisfying $c_1$ would satisfy $c_0$.
%
In other words, we need to prove that there is no input that satisfies $c_1$ and does not satisfy $c_0$.
%
So we feed the logical constrains $c_1 \land \neg c_0$ in the SMT solver. 
%
The SMT solver returns \textbf{unsat}, which means this logical constrains is unsatisfiable.
%
It indicates there is no such input that satisfies $c_1$ and does not satisfy $c_0$.

%
In the second step, we need to prove for the same input, if both program $p_0$ and $p_1$ reach the return statement,
then they return the same vector.
%
In the other words, we need to prove that there is no input, such that $p_0$ and $p_1$ return different vectors.
%
Because $p_0$ and $p_1$ both directly return the input vector, this statement holds trivially.
%
We can also encode the equivalent of the output as a logic constrains $Equal := SEMPNO = SEMPNO \land SENAME = 
SENAME \land  SDEPTNO \land SDEPTNO$.
%
Then we feed the constrains $\neg Equal \land c_1 \land c_0$ , which represent both programs reach the return 
statement and return different vector, 
into the SMT solver.
%
The SMT solver returns \textbf{unsat}, which indicates there is no such input for which the two programs return two 
different vectors.
%
Combing the proof that if $p_1$ reaches the return statement, then $p_0$ reaches the return statement.
%
We prove SQL query $q_0$ contains $q_1$.
%
By proving $q_1$ contains $q_0$ by the same approach, we prove $q_0$ is semantically equivalent to $q_1$.



\AJ{Problem formalization should come after overview.}
\input{formalize}

% bibliography:
\newcommand{\newblock}{}
\bibliographystyle{abbrv}
\small

\bibliography{p}


\end{document}
