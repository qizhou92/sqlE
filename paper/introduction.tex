\section{Introduction}\label{sec:introduction}
% motivation

The proliferation of cloud computing has resulted in the availability of
a growing number of database-as-a-service (DBaaS) offerings (e.g., 
Microsoft's Azure Data Lake, Google's BigQuery). 
These DBaaS solutions allow end users to examine data through on-demand
services, without the need to first install any hardware or software on their
machines. They offer multiple benefits over traditional on-prem DBMSs,  
including lower software licensing and infrastructure costs, 
rapid provisioning, 
reduced infrastructure management overhead, 
ability to elastically scale resources to meet demand, and 
higher availability.

DBaaS solutions enable end users to quickly create and deploy complex data
processing tasks. Prior work has shown that these tasks may have significant
overlap of computation (i.e., redundant execution of certain sub-tasks).
For example, around 40\% of the tasks executed on Microsoft's SCOPE 
service have computation overlap with other tasks. This results in 
increased consumption of computational resources, higher data processing costs,
and longer task execution times. 

Developers and database administrators may resolve these problems by increasing
the modularity of their data processing tasks to reuse of the results of
frequently executed sub-tasks (i.e., queries). 
However, in practice, it is challenging for developers and database
administrators (DBAs) to manually detect overlap across tasks since they 
may be distributed across teams, organization roles,  and geographic locations.
Thus, we require automated cloud-scale tools for identifying 
semantically equivalent queries.

The fundamental problem of determining if two SQL queries are semantically
equivalent or not is undecidable~\cite{abiteboul95}.
Given this constraint, researchers have focused on identifying a subset of
relational algebra where it is feasible to determine equivalence of queries
under set and bag semantics\footnote{define}\AJ{citations}. This line of
research examined the theoretical underpinnings of this problem by targeting
conjunctive queries. This limits their ability to identify overlap in 
real-world SQL queries.

A more pragmatic approach to query equivalence has been proposed by the 
recently developed \textbf{COSETTE} and \textbf{UDP} tools.
They transform SQL queries to an algebraic structure and use a decision
procedure to compare the resulting algebraic expressions.
These tools determine query equivalence by finding isomorphisms and
homomorphisms between the algebraic expressions under the set or bag
definitions.
\textbf{COSETTE} and \textbf{UDP} \QZ{cite} uses different algebraic structures
for transforming SQL queries. While the former tool uses \textbf{k-relations},
the latter leverages \textbf{unbounded semiring}\AJ{Why different structures}.

% why it is bad
While these tools have been able to prove equivalence of real-world SQL queries,
they suffer from two limitations.
First, they cannot model the semantics of many widely-used aspects of SQL
queries including arithmetic operations, NULL values, and complex query
predicates. This restricts their usability on complex real-world queries.
Second, their decision procedures are computationally intensive and
may not be suitable for analysing cloud-scale workloads.

% contribution
The first contribution of this paper is proposing to use SMT solver, which is a
widely used technique in programming verification community, to decide the
semantically equivalence of sql queries under \textbf{set} definition.
This approach views each SQL query as a loop-free and guaranteed terminated
program.
This approach allows us to use logic constrains to encode the semantic of large
subset of SQL features, including filter operation with all common used
predicate, extended projection with complex arithmetic operations, three
different type of join operation, union operation, aggregate operation and NULL
values.
Because this problem is undecidable, this approach uses over-approximation to be
sound but incomplete.

The second contribution is an implemented tool \sys that can automatically
decide the equivalence of sql queries under set definition.
Because deciding the semantical equivalence of a given pair of SQL queries is an
undecidable problem, \sys is sound but incomplete.
Sound implies if \sys gives an equivalent decision, the given pair of SQL
queries are semantically equivalent.
By Incomplete we mean that if \sys gives an inequivalent decision, the given
pair of SQL queries might be indeed semantically equivalent. In other words, our
system \sys may result in some false negatives.
We apply \sys to an open benchmark set from calcite\QZ{cite}.
The evaluation result shows that \sys can prove semantical equivalence for more
pairs of sql queries than \textbf{UDP} under set definition.
We also apply \sys to a large query dataset.
The result shows that there are over 10 percent real word queries currently in
use at a reputed company xxxxx that contains at least one subquery is
semantically equivalent or be contained by other queries' subquery, which
indicates a big opportunity for optimization.

% Paper outline.
The rest of this paper is organized as follows.
\autoref{sec:formalize} formally defines the target SQL query and the problem.
\autoref{sec:overview} use a simple example to demonstrate the whole system.
and \autoref{sec:approach} presents the \sys in details.
\autoref{sec:evaluation} presents an empirical evaluation of \sys.
\autoref{sec:related-work} compares our contributions to related work, and
\autoref{sec:conclusion} concludes.
