\section{Overview}\label{sec:overview}
%
In this section, we demonstrate our approach by an example.
%
We first introduce a pair of SQL queries, their corresponding normalized representation and the table schema.
%
Then we illustrate how to prove their semantic equivalence by our system.
%
Finally, we introduce a key technique SMT (Satisfiability Modulo Theory) solver that our system is based on.

%
\begin{figure}
\input{code/q1.sql}
\caption{%
    A pair of sql query from calcite.
  }\label{fig:q1}
\end{figure}
%

\autoref{fig:q1} contains a pair of SQL queries, their corresponding normalized representation and the table schema.
%
$EMP$ is a table that has three columns $EMPNO$, $ENAME$ and $DEPTNO$.
%
$EMPNO$ and $DEPTNO$ are integer type, and $ENAME$ is string type.
%
The query $q_0$ first selects all tuples from table $EMP$ that $DEPTNO$ is equal to
10 and calls the result as $t$.
%
Then $q_0$ selects all tuples from $t$ such that $DEPTNO$ plus five is larger than $EMPNO$.
%
The query $q_1$ first selects all tuples from table $EMP$ that $DEPTNO$ is equal to
ten and calls the result as $t1$.
%
Then $q_1$ selects all tuples from $t1$ such that $EMPNO$ is less than 15.
%
These two sql queries are obviously semantically equivalent 
because once the $DEPTNO$ is equal to ten, $DEPTNO$ plus five is equal to 15.
%
In order to prove the semantic equivalence of $q_0$ and $q_1$, we convert $q_0$ and 
$q_1$ into the syntax we defined in \autoref{sec:formalize}.
%
The relation, that returns by executing $q_0$ on a valid input $T$, is denoted as $R_0$.
%
The relation that returns by executing $q_1$ is denoted as $R_1$.
%
In this example, because the only related table is $EMP$, the valid input $T$ is the table $EMP$ with a
finite number of tuples.

%
Based on the definition of equivalence in \autoref{sec:formalize}, in order to prove $q_0$ is semantically
equivalent to $q_1$, we need to prove $q_0$ contains $q_1$ and $q_1$ contains $q_0$.
%
We use $q_0$ contains $q_1$ as an example to demonstrate how to verify containment of sql queries.
%
Based on the definition of containment in \autoref{sec:formalize}, in order to prove $q_0$ contains 
$q_1$, we need to prove for all valid inputs $T$, all vectors, that are $R_1$, are also in $R_0$. 
%
In other words, we need to prove for all valid input, there is no vector that in $R_1$ is not in $R_0$.
%

% the intuition of the design
If we consider all vectors in $R_0$ and $R_1$, each of them is actually directly from a tuple in the input table $EMP$
because of the semantic of filter operation.
%
Furthermore, if we only consider sql query that contains table, filter, project and inner join operation, 
then each vector in the final relation is constructed by choosing one tuple
from each table as long as the query does not contain a table with duplicate tuples.
%
A query that contains duplicate tuples will be considered later in \QZ{section}.
%
Thus this type of query can be viewed as a multiple execution of a program that takes one vector
from each table that either constructs a new vector in the final relation, or does not return any vector.
%
In this example, $q_0$ and $q_1$ can be viewed as multiple execution of the program $p_0$ and $p_1$ that is shown in \autoref{fig:q2}
%
\begin{figure}
\input{code/q2.java}
\caption{%
    programs that construct vector.
  }\label{fig:q2}
\end{figure}
%
$p_0$ and $p_1$ takes a vector from the input table $EMP$ and outputs a vector if it satisfies certain conditions.
%
If we can prove the following statement for any given input, which is a vector of table $EMP$, then we prove that $q_0$ contains $q_1$.
%
The statement is for the same input, if program $p_1$ returns a vector, then program $p_0$ returns the same vector.
%
Proving such statement needs two steps.
%
First we need to prove for an arbitrary input vector $EMPNO,ENAME,DEPTNO$, if $p_1$ reaches the return statement, which means this vector is not 
eliminated by the filter operation of $q_1$, then $p_0$ also reaches the return statement.
%
Then we need to prove for an arbitrary input vector that $p_1$ and $p_0$ both reach the return statement, and also that $p_1$ and $p_0$ returns the same vector.
%
This statement holds trivially for this example, because both sql queries directly return the input vector.
%
But for queries that contain project operation, we need to verify the output vectors are the same by two queries.

%how to do it.
The key here is that program $p_0$ and $p_1$ are loop-free due to the nature of SQL queries and are guaranteed to terminate.
%
Verifying this property for loop-free and guaranteed terminated programs is a well-studied problem in program language community.
%
Using satisfiability modulo theories(SMT) solver is a widely accepted technique in program verification problems.
%
We translate queries into logic constrains, and use SMT solver to decide if these constrains are satisfiable to judge if one query contains
the other.

%
In this example, because the input is a vector of three elements, we create three variables $SEMPNO$,$SENAME$ and $SDEPTNO$ in the logic constrains that each 
corresponding an element in the input vector.
%
The condition of program $p_0$ reaches the return statement can be encoded into an first order logic constrains \\
$c_0 := SDEPTNO = 10 \land SDEPTNO + 5 > SEMPNO$.
%
And the logic constrains for program $p_1$ reaches the return statement is $c_1 := SDEPTNO = 10 \land 15 > SEMPNO$.
%
For verifying $p_1$ reaches the return statement implies $p_0$ reaches the return statement, 
we need to verify for any input that satisfying $c_1$ would satisfy $c_0$.
%
In other words, we need to prove that there is no input that satisfies $c_1$ and does not satisfy $c_0$.
%
So we feed the logical constrains $c_1 \land \neg c_0$ in the SMT solver. 
%
The SMT solver returns \textbf{unsat}, which means this logical constrains is unsatisfiable.
%
It indicates there is no such input that satisfies $c_1$ and does not satisfy $c_0$.

%
In the second step, we need to prove for the same input, if both program $p_0$ and $p_1$ reach the return statement, then they return the same vector.
%
In the other words, we need to prove that there is no input, such that $p_0$ and $p_1$ return different vectors.
%
Because $p_0$ and $p_1$ both directly return the input vector, this statement holds trivially.
%
We can also encoded the equivalent of the output as a logic constrains $Equal := SEMPNO = SEMPNO \land SENAME = SENAME \land  SDEPTNO \land SDEPTNO$.
%
Then we feed the constrains $\neg Equal \land c_1 \land c_0$ , which represent both programs reach the return statement and return different vector, 
into the SMT solver.
%
The SMT solver returns \textbf{unsat}, which indicates there is no such input for which the two programs return two different vectors.
%
Combing the proof that if $p_1$ reaches the return statement, then $p_0$ reaches the return statement.
%
We prove SQL query $q_0$ contains $q_1$.
%
By proving $q_1$ contains $q_0$ by the same approach, we prove $q_0$ is semantically equivalent to $q_1$.

%summary, introduce basic structure of the system
In short summary, 

