\documentclass{vldb}

\newcommand{\paperTitle}{Proving Equivalence of SQL Queries Using Satisfiability
Modulo Theories} 
\newcommand{\paperKeywords}{Query Rewriting, Query Optimization, Satisfiability
Modulo Theories}
\newcommand{\paperAuthors}{}

% Define to fix sigplanconf.
% \doi{}

\setlength{\paperheight}{11in}
\setlength{\paperwidth}{8.5in}

\usepackage[square,comma,numbers,sort&compress]{natbib}
%\usepackage[hyphens]{url}
\usepackage{hyperref}

% hyperref itself
\hypersetup{
 pdfauthor = {\paperAuthors}, 
 pdftitle = {\paperTitle}, 
 pdfkeywords = {\paperKeywords}, 
 pdfborder={ 0 0 0 }
}

\usepackage[usenames,dvipsnames]{xcolor}
\usepackage{amsmath,amsopn,amssymb}
\usepackage{subfig}
\usepackage{endnotes,microtype,xspace,graphicx,fancyvrb,multirow}
\usepackage{booktabs}
\usepackage{array,underscore,relsize}
\usepackage[T1]{fontenc}
\usepackage{times}
\usepackage{fancyhdr,lastpage}
\usepackage{enumitem}
% \usepackage[labelfont=bf,font=small,skip=5pt]{caption}

%\usepackage{lmodern}

% aliascnt: counter stuff that works with theorem environments.
\usepackage{aliascnt}

% algorithm2e: algorithms
\usepackage[linesnumbered]{algorithm2e}

% semantic: inference rules
\usepackage{semantic}

% stmaryrd: more math fonts (namely mathbb)
\usepackage{stmaryrd}

% floatrow: putting two figures side by side
\usepackage{floatrow}

\usepackage{mathtools}

\pagestyle{fancy}
\fancyhf{}
\renewcommand{\headrulewidth}{0pt}
\cfoot{\thepage}

\newcommand{\sys}{\mbox{\textsc{Shara}}\xspace}

% cmds: typesetting commands
\input{cmds}

% std: standard math commands and environments.
% Standard math shorthand.
\newcommand{\add}[2]{#1 \union \{ #2 \}}

\newcommand{\assign}{\mathbin{:=}}

\newcommand{\auflia}{\textsc{Auflia}\xspace}

\newcommand{\bigland}{\bigwedge}

\newcommand{\biglor}{\bigvee}

\newcommand{\bigunion}{\bigcup}

\newcommand{\bools}{\mathbb{B}}

\newcommand{\bv}{\textsc{Bv}\xspace}

\newcommand{\code}{\mathtt}

\newcommand{\compose}{\circ}

\newcommand{\concat}{\cdot}

\newcommand{\cons}{\mathbin{::}}

\newcommand{\disjunion}{\dot{\cup}}

\newcommand{\domain}{\mathsf{Dom}}

\newcommand{\elts}[1]{\{ #1 \}}

\newcommand{\entails}{\models}

\newcommand{\euf}{\textsc{Euf}\xspace}

\newcommand{\false}{\mathsf{False}}

\newcommand{\intersection}{\cap}

\newcommand{\ints}{\mathbb{Z}}

\newcommand{\lia}{\textsc{Lia}\xspace}

\newcommand{\nats}{\mathbb{N}}

\newcommand{\none}{\mathsf{None}}

\newcommand{\partto}{\hookrightarrow}

\newcommand{\pset}{\mathcal{P}}

\newcommand{\range}{\mathsf{Rng}}

\newcommand{\remove}[2]{#1 \setminus \{ #2 \}}

\newcommand{\replace}[3]{#1 [ #3 / #2 ]}

\newcommand{\restrict}[2]{#1|_{#2}}

\newcommand{\sats}{\vdash}

\newcommand{\setformer}[2]{\{ #1\ |\ #2 \}}

\newcommand{\some}{\mathsf{Some}}

\newcommand{\subs}[2]{#1 [ #2 ]}

\newcommand{\tc}[1]{#1^{*}}

\newcommand{\true}{\mathsf{True}}

\newcommand{\undef}{\uparrow}

\newcommand{\union}{\cup}

\newcommand{\upd}[3]{#1[#2 \mapsto #3]}

% Theorem environments
\newtheorem{cor}{\bf{Corollary}}

\newtheorem{defn}{\bf{Definition}}

\newtheorem{ex}{\bf{Example}}

\newtheorem{lem}{\bf{Lemma}}

\newtheorem{thm}{\bf{Theorem}}

% Register classes of stuff to refer to with autoref.
\renewcommand{\algorithmautorefname}{Alg.}

\newcommand{\appautorefname}{App.}

\newcommand{\corautorefname}{Cor.}

\newcommand{\defnautorefname}{Defn.}

\newcommand{\exautorefname}{Ex.}

\newcommand{\figautorefname}{Fig.}

\newcommand{\lemautorefname}{Lemma}

\newcommand{\lineautorefname}{Line}

\newcommand{\thmautorefname}{Thm.}

%%% Local Variables: 
%%% mode: latex
%%% TeX-master: "p"
%%% End: 


% shorthand: paper-specific shorthand commands.
% shorthand: paper-specific shorthand
\newcommand{\apps}[1]{\mathsf{Apps}_{#1}}

\newcommand{\appsof}[1]{\mathsf{Apps}[ #1 ]}

\newcommand{\argsof}[1]{\mathsf{Args}[ #1 ]}

\newcommand{\arityof}[1]{\mathsf{Arity}_{#1}}

\newcommand{\bodies}[1]{\mathsf{Bodies}_{#1}}

\newcommand{\bodyof}[1]{\mathsf{Body}_{#1}}

\newcommand{\calloneref}{\textbf{(5)}\xspace}

\newcommand{\callzeroref}{\textbf{(2)}\xspace}

\newcommand{\chc}[1]{\mathsf{CHC}[ #1 ]}

\newcommand{\chcs}[1]{\mathsf{CHCs}[ #1 ]}

\newcommand{\clauses}[1]{\mathsf{Clauses}[ #1 ]}

\newcommand{\collapse}[2]{\textsc{Collapse}_{#1}^{#2}}

\newcommand{\copyrel}[1]{\textsc{CopyRel}_{#1}}

\newcommand{\corr}[1]{\textsc{Corr}_{#1}}

\newcommand{\ctrof}[1]{\mathsf{Ctr}[#1]}

\newcommand{\ctxrelof}[1]{\mathsf{CtxRel}[ #1 ]}

% array indexing: confirmed
\newcommand{\ctxs}[1]{\mathsf{Ctxs}[ #1 ]}

\newcommand{\dblref}{\textbf{(1)}\xspace}

\newcommand{\degreeof}[1]{\mathsf{deg}(#1)}

\newcommand{\deps}{\textsc{Deps}\xspace}

\newcommand{\depsof}[1]{\mathsf{Deps}_{#1}}

\newcommand{\tdepsof}[1]{\mathsf{TrDeps}_{#1}}

\newcommand{\domainof}[1]{\mathsf{Dom}[ #1 ]}

\newcommand{\duality}{\textsc{Duality}\xspace}

\newcommand{\edgelblof}[1]{\mathsf{EdgeLbl}[ #1 ]}

\newcommand{\eldarica}{\textsc{Eldarica}\xspace}

\newcommand{\elseref}{\textbf{(4)}\xspace}

\newcommand{\expand}[1]{\textsc{Expand}_{#1}}

\newcommand{\expandaux}{\textsc{ExpAux}\xspace}

\newcommand{\expandsto}{\preceq}

\newcommand{\formulas}{\mathsf{Forms}}

\newcommand{\headof}[1]{\mathsf{Head}_{#1}}

\newcommand{\siblingof}[1]{\mathsf{Siblings}_{#1}}

\newcommand{\impact}{\textsc{Impact}\xspace}

\newcommand{\instances}[1]{\mathsf{Instances}[ #1 ]}

\newcommand{\interps}[1]{\mathsf{Interps_{ #1 }}}

\newcommand{\issat}{\textsc{IsSat}\xspace}

\newcommand{\mcchc}{\mathcal{S}_{\cc{MC}}}

\newcommand{\mcpreds}{\mathcal{R}[ \cc{MC} ]}

\newcommand{\mcsolve}{\textsc{SolveCdd}[ \cc{MC} ]}

\newcommand{\nodelblof}[1]{\mathsf{NodeLbl}[ #1 ]}

\newcommand{\nodesof}[1]{\mathsf{Nodes}[ #1 ]}

\newcommand{\nosoln}{\mathsf{None}}

\newcommand{\postctr}[1]{\textsc{Post}_{#1}}

\newcommand{\prectr}[1]{\textsc{Pre}_{#1}}

\newcommand{\predatomsof}[1]{\mathsf{Atoms}_P[ #1 ]}

\newcommand{\predof}[1]{\mathsf{Pred}[ #1 ]}

\newcommand{\queryof}[1]{\mathsf{Query}_{#1}}

\newcommand{\relof}[1]{\mathsf{Rel}[ #1 ]}

\newcommand{\relpreds}{\mathsf{Preds}}

\newcommand{\relsof}[1]{\mathsf{Rels}_{#1}}

\newcommand{\relvarsof}[1]{\mathsf{RelVars}[ #1 ]}

\newcommand{\seahorn}{\textsc{SeaHorn}\xspace}

\newcommand{\shara}[1]{\textsc{Shara}_{#1}}

\newcommand{\sharingclause}{\textsc{SharedRel}\xspace}

\newcommand{\siblings}[1]{\mathsf{Siblings}[ #1 ]}

\newcommand{\singleton}[1]{\textsc{Single}[ #1 ]}

\newcommand{\sinkof}[1]{\mathsf{sink}[ #1 ]}

\newcommand{\solveaux}{\textsc{SAux}\xspace}

\newcommand{\solvecdd}[1]{\textsc{SolveCdd}_{#1}}

\newcommand{\solveitp}{\textsc{Itp}\xspace}

\newcommand{\srcof}[1]{\mathsf{src}[ #1 ]}

\newcommand{\tformulas}[1]{\mathsf{Forms}[ #1 ]}

\newcommand{\thenref}{\textbf{(3)}\xspace}

\newcommand{\unit}{\mathsf{unit}}

\newcommand{\used}[1]{\mathsf{Use}[ #1 ]}

\newcommand{\vars}{\cc{Vars}}

\newcommand{\varsof}[1]{\mathsf{Vars}_{#1}}

\newcommand{\vc}[1]{\textsc{Cex}_{#1}}

\newcommand{\vinta}{\textsc{Vinta}\xspace}

\newcommand{\vocab}{\mathsf{Vocab}}

\newcommand{\whale}{\textsc{Whale}\xspace}

\newcommand{\zthree}{\textsc{Z3}\xspace}

%%% Local Variables: 
%%% mode: latex
%%% TeX-master: "p"
%%% End: 

\newcommand{\sql}{\textsf{Q}}
\newcommand{\STable}[1]{\textsf{Table}~#1}
\newcommand{\SFilter}[2]{\textsf{Filter}~#1~#2}
\newcommand{\SProj}[2]{\textsf{Proj}~#1~#2~}
\newcommand{\SJoin}[4]{\textsf{Join}~#1~#2~#3~#4}
\newcommand{\SUnion}[2]{\textsf{Union}~#1~#2~}
\newcommand{\SAggregate}[3]{\textsf{Aggregate}~#1~#2~#3}
\newcommand{\evaluation}[3]{\langle #1 \models #2 \rangle \Downarrow #3}
\newtheorem{mydef}{Definition}
% rev: revision information: generated by build system
\input{rev}
\vldbTitle{A Sample Proceedings of the VLDB Endowment Paper in LaTeX Format}
\vldbAuthors{Ben Trovato, G. K. M. Tobin, Lars Th{\sf{\o}}rv{$\ddot{\mbox{a}}$}ld, Lawrence P. Leipuner, Sean Fogarty, Charles Palmer, John Smith, Julius P.~Kumquat, and Ahmet Sacan}
\vldbDOI{https://doi.org/10.14778/xxxxxxx.xxxxxxx}
\vldbVolume{12}
\vldbNumber{xxx}
\vldbYear{2018}

\title{\paperTitle}
\author{\paperAuthors}

\begin{document}

\maketitle

\begin{abstract}

The design of the buffer manager in database management systems (DBMSs)
is influenced by the performance characteristics of volatile
memory (DRAM) and non-volatile storage (e.g., SSD).
The key design assumptions have been that the data must be migrated to DRAM 
for the DBMS to operate on it and that storage is orders of magnitude slower
than DRAM. But the arrival of new non-volatile memory (NVM) technologies that
are nearly as fast as DRAM invalidates these previous assumptions.

This paper presents techniques for managing and designing a multi-tier storage
hierarchy comprising of DRAM, NVM, and SSD.
Our main technical contributions are a multi-tier buffer manager and a storage
system designer that leverage the characteristics of NVM.
We propose a set of optimizations for maximizing the utility of data migration
between different devices in the storage hierarchy. 
We demonstrate that these optimizations have to be tailored based on 
device and workload characteristics. 
Given this, we present a technique for adapting these optimizations to achieve a
near-optimal buffer management policy for an arbitrary workload and storage
hierarchy without requiring any manual tuning.
We finally present a recommendation system for designing a multi-tier storage
hierarchy for a target workload and system cost budget.
Our results show that the NVM-aware buffer manager and storage system designer
improve throughput and reduce system cost across different transaction and
analytical processing workloads.
\end{abstract}

% abstract.tex: DEP, contains abstract
\section{Introduction}\label{sec:introduction}
% motivation

The proliferation of cloud computing has resulted in the availability of
a growing number of database-as-a-service (DBaaS) offerings (e.g., 
Microsoft's Azure Data Lake, Google's BigQuery). 
These DBaaS solutions allow end users to examine data through on-demand
services, without the need to first install any hardware or software on their
machines. They offer multiple benefits over traditional on-prem DBMSs,  
including lower software licensing and infrastructure costs, 
rapid provisioning, 
reduced infrastructure management overhead, 
ability to elastically scale resources to meet demand, and 
higher availability.

DBaaS solutions enable end users to quickly create and deploy complex data
processing tasks. Prior work has shown that these tasks may have significant
overlap of computation (i.e., redundant execution of certain sub-tasks).
For example, around 40\% of the tasks executed on Microsoft's SCOPE 
service have computation overlap with other tasks. This results in 
increased consumption of computational resources, higher data processing costs,
and longer task execution times. 

Developers and database administrators may resolve these problems by increasing
the modularity of their data processing tasks to reuse of the results of
frequently executed sub-tasks (i.e., queries). 
However, in practice, it is challenging for developers and database
administrators (DBAs) to manually detect overlap across tasks since they 
may be distributed across teams, organization roles,  and geographic locations.
Thus, we require automated cloud-scale tools for identifying 
semantically equivalent queries.

The fundamental problem of determining if two SQL queries are semantically
equivalent or not is undecidable~\cite{abiteboul95}.
Given this constraint, researchers have focused on identifying a subset of
relational algebra where it is feasible to determine equivalence of queries
under set and bag semantics\footnote{define}\AJ{citations}. This line of
research examined the theoretical underpinnings of this problem by targeting
conjunctive queries. This limits their ability to identify overlap in 
real-world SQL queries.

A more pragmatic approach to query equivalence has been proposed by the 
recently developed \textbf{COSETTE} and \textbf{UDP} tools.
They transform SQL queries to an algebraic structure and use a decision
procedure to compare the resulting algebraic expressions.
These tools determine query equivalence by finding isomorphisms and
homomorphisms between the algebraic expressions under the set or bag
definitions.
\textbf{COSETTE} and \textbf{UDP} \QZ{cite} uses different algebraic structures
for transforming SQL queries. While the former tool uses \textbf{k-relations},
the latter leverages \textbf{unbounded semiring}\AJ{Why different structures}.

% why it is bad
While these tools have been able to prove equivalence of real-world SQL queries,
they suffer from two limitations.
First, they cannot model the semantics of many widely-used aspects of SQL
queries including arithmetic operations, NULL values, and complex query
predicates. This restricts their usability on complex real-world queries.
Second, their decision procedures are computationally intensive and
may not be suitable for analysing cloud-scale workloads.

% contribution
The first contribution of this paper is proposing to use SMT solver, which is a
widely used technique in programming verification community, to decide the
semantically equivalence of sql queries under \textbf{set} definition.
This approach views each SQL query as a loop-free and guaranteed terminated
program.
This approach allows us to use logic constrains to encode the semantic of large
subset of SQL features, including filter operation with all common used
predicate, extended projection with complex arithmetic operations, three
different type of join operation, union operation, aggregate operation and NULL
values.
Because this problem is undecidable, this approach uses over-approximation to be
sound but incomplete.

The second contribution is an implemented tool \sys that can automatically
decide the equivalence of sql queries under set definition.
Because deciding the semantical equivalence of a given pair of SQL queries is an
undecidable problem, \sys is sound but incomplete.
Sound implies if \sys gives an equivalent decision, the given pair of SQL
queries are semantically equivalent.
By Incomplete we mean that if \sys gives an inequivalent decision, the given
pair of SQL queries might be indeed semantically equivalent. In other words, our
system \sys may result in some false negatives.
We apply \sys to an open benchmark set from calcite\QZ{cite}.
The evaluation result shows that \sys can prove semantical equivalence for more
pairs of sql queries than \textbf{UDP} under set definition.
We also apply \sys to a large query dataset.
The result shows that there are over 10 percent real word queries currently in
use at a reputed company xxxxx that contains at least one subquery is
semantically equivalent or be contained by other queries' subquery, which
indicates a big opportunity for optimization.

% Paper outline.
The rest of this paper is organized as follows.
\autoref{sec:formalize} formally defines the target SQL query and the problem.
\autoref{sec:overview} use a simple example to demonstrate the whole system.
and \autoref{sec:approach} presents the \sys in details.
\autoref{sec:evaluation} presents an empirical evaluation of \sys.
\autoref{sec:related-work} compares our contributions to related work, and
\autoref{sec:conclusion} concludes.


\AJ{Need to break down approach into three sections similar to UDP paper. Example:
\begin{enumerate}[noitemsep,nolistsep,label=\protect\BC{\arabic*},leftmargin=7ex]
\setlength{\itemsep}{-0pt}
\item Axiomatic Foundations (?)
\item Aggregates etc.
\item Decision Procedures (?)
\end{enumerate}
}

\section{Overview}\label{sec:overview}
%
In this section, we demonstrate our approach by two examples.
%
We first introduce two pairs of semantically equivalent SQL queries and the table schema from the open database
framework calcite\QZ{cite}.
%
Then we show the fundamental reasons that the previous approach cannot prove their equivalence.
%
Finally, we demonstrate how our approach applies Satisfiability Modulo Theory (SMT) solver to prove the equivalence 
of these two examples. 
%
\begin{figure}
\input{code/q1.sql}
\caption{%
    A pair of sql queries from calcite.
  }\label{fig:q1}
\end{figure}
%

%introduce the first query
\autoref{fig:q1} contains a pair of SQL queries and its table schema.
%
$EMP$ is a table that has three columns $EMPNO$, $ENAME$ and $DEPTNO$.
%
$EMPNO$ and $DEPTNO$ are integer type, and $ENAME$ is string type.
%
The query $q_0$ first selects all tuples from table $EMP$ that $DEPTNO$ is equal to
10 and calls the result as $t$.
%
Then $q_0$ selects all tuples from $t$ such that $DEPTNO$ plus five is larger than $EMPNO$.
%
The query $q_1$ first selects all tuples from table $EMP$ that $DEPTNO$ is equal to
ten and calls the result as $t1$.
%
Then $q_1$ selects all tuples from $t1$ such that $EMPNO$ is less than 15.
%
These two sql queries are obviously semantic equivalent 
because once the $DEPTNO$ is equal to ten, $DEPTNO$ plus five equals to 15.
%


%introduce the second query
%
\begin{figure}
\input{code/q2.sql}
\caption{%
    A pair of sql queries from calcite.
  }\label{fig:q2}
\end{figure}
%
\autoref{fig:q2} contains a pair of SQL queries with the same table schema as above.
%
The query $q_0$ selects all tuples from table $EMP$ such that $EMPNO$ equals to 10, and
$EMPNO$ is not NULL.
%
The query $q_1$ selects all tuples from table $EMP$ such that $EMPNO$ equals to 10.
%
These two sql queries are semantically equivalent. 
%
Because based on three value logic, if $EMPNO$ equals to 10 be evaluated as true, then 
$EMPNO$ is not NULL.  

%talk about the previous approach
Previous approaches such as Cosette\QZ{cite} and Udp\QZ{cite} fails to prove the semantic
equivalence of these two pairs of queries.
%
Cosette and Udp use K-relations and U-semiring respectively as a simple algebra structure for
representing sql queries.
%
After sql queries being translated into nominalized form in a simple algebra strcutre, then Cosette and
Udp applies a set of rewrite rules on algebra expression.
%
If an isomorphism can be found between two rewrited algebra expressions, then the two sql queries are equivalent.

%Why it fails for the first example
For proving the equivalence of the pair of queries in \autoref{fig:q1}, two queries can be 
represented as the following algebra representation.
%
\[q1 : \lbrack t.DEPTNO = 10 \rbrack \times \lbrack t.DEPTNO + 5 > t.EMPNO \rbrack \times EMP(t)\]
%
\[q2:  \lbrack t.DEPTNO = 10 \rbrack \times \lbrack 15 > t.EMPNO \rbrack \times EMP(t)] \]
%
The previous approaches cannot prove these two queries are equivalent.
%
Because this proof requires modeling the semantic of arithmetic operations and combining the semantic of
each predicate.
%
The automatic proof system needs to be able to infer the fact that the predicate 
$\lbrack t.DEPTNO + 5 > t.EMPNO \rbrack$ and predicate $\lbrack 15 > t.EMPNO$ are identical if the predicate
$\lbrack t.DEPTNO = 10 \rbrack$ holds.


%Why it fails for the second example
For proving the equivalence of the pair of queries in \autoref{fig:q2}, two queries can be 
represented as the following algebra representation.
%
\[q1 : \lbrack t.DEPTNO = 10 \rbrack \times \lbrack NOT\_NULL(t.DEPTNO) \rbrack \times EMP(t)\]
%
\[q2:  \lbrack t.DEPTNO = 10 \rbrack \times EMP(t)] \]
%
The previous approaches cannot prove these two queries are equivalent.
%
Because this proof requires modeling the three logic system for null value, modeling the semantic of 
equal predicates and combining the semantic of each predicate.
%
The automatic proof system needs to be able to infer that if $\lbrack t.DEPTNO = 10 \rbrack$ holds, 
then $t.DEPTNO$ cannot be an NULl value.
%
The fact needs to be inferred is non-trivial.
%
Because for some binary predicates, the column can be NULL when the predicate be evaluated as
true.
%
For example, predicate $\lbrack OR TRUE t.DEPTNO \rbrack$ holds in three logic system when $t.DEPTNO$ is NULL.

% key reasons why it fails
There are two main reasons that the approaches of translating SQL queries into simple algebra expressions 
and deciding the equivalence by applying re-write rule to find isomorphism are limited.
%
First, as the above example, each predicate has certain sematically meaning that would affect other predicates.
%
These approaches unable to modeling the semantic meaning of all predicates and combining these semantic of these
predicates.
%
Second, not only predicates, but also some operator has certain semantically would affect other predicates.
%
For example, for the left outer join, when the tuples in left table fails to find the matching tuples 
in the right tuples, the new tuples is concatenation of left tuple with null values.
%
If $NOT\_NULL$ predicates be applied to the new tuples, it eliminates all tuples that columns from 
right table are null. 
%
The left outer join actually be reduced to inner join when certain predicates applied.


%
In order to prove the semantic equivalence of queries in \autoref{fig:q1} and \autoref{fig:q2}.
%
We proposed use satisfiability modulo theories(SMT) solver, which is 
a widely accepted technique in program verification problems, to verifying the equivalence between SQL queries.
%

We prove the semantic equivalence of SQL queries under \textbf{set} definition.
%
It means two SQL queries are be considered semantic equivalent if and only if for all valid input tables, 
the output tables are the same of applying these two queries on the same input tables and eliminating all duplicates
tuples in the output tables.
%
We choose the semanticaly equivalence under the \textbf{set} definition becuase \QZ{related to motivation}.
%
We prove the equivalence between two queries by proving two containment relations that each query contains the other.
%
We define containment relations under set definition as query $q_1$ contains query $q_2$ if and only if
for all valid input tables, the tuples in the output table of applying $q_2$ on input tables all shows in
the output table of applying $q_1$ on the same input tables.
%
In other words, if there is no tuple in the output table of $q_2$ that not int output table of $q_1$ for the same
valid inputs, then $q_1$ contains $q_2$.
%
The formation definitions of containment and equivalent are given in \autoref{sec:formalize}.

%
We have reduced the problem of verifying the equivalence of sql queries to verify the containment relations of sql
queries.
%
We use pairs of queries in \autoref{fig:q1} and \autoref{fig:q2} as examples to demonstrate how to verify 
containment relations of sql queries.
%

% the intuition of the design
The key observation for queries in \autoref{fig:q1} and \autoref{fig:q2} is that each tuple in the output table is
corresponding a tuple in the original input table $EMP$.
%
The semantic of these queries is filtering out tuples in table $EMP$ that does not satisfy a set of predicates.
%
Furthermore, if we only consider sql query that contains table, filter, project and inner join operation, 
then each tuples in the output table is constructed by choosing one tuple
from each input table.
%
This type of queries can be viewed as applying a program, that takes one tuple
from each input table that returns a new tuple in the output table if certain predicates are satisfied, on the 
cartesian product of all input tables.
%
Queries that contains left outer join, full outer join, union and aggregate will be discussed in \QZ{section}.

%
In this example, $q_1$ and $q_2$ in \autoref{fig:q1} can be viewed as multiple execution of the program 
$p_0$ and $p_1$ that is shown in \autoref{fig:p1}
%
\begin{figure}
\input{code/p1.java}
\caption{%
    programs that represent queries.
  }\label{fig:p1}
\end{figure}
%
$p_1$ and $p_2$ takes a tuple from the input table $EMP$ and outputs a tuple if it satisfies certain conditions.
%
We can prove query $q_1$ contains $q_2$ by proving following statement:
For a given input which is a tuple from $EMP$, if program $p_2$ returns a tuple, 
then $p_1$ returns the same tuple.

%
Proving this statement needs two steps.
%
First we need to prove for an arbitrary input tuple $EMPNO,ENAME,DEPTNO$, if program $p_2$ 
reaches the return statement, then program $p_1$ also reaches the return statement.
%
Then we need to prove for an arbitrary input tuple that program $p_1$ and $p_2$ both reach the return statement, 
program $p_1$ and $p_0$ returns the same tuple.
%
This condition holds trivially for this example, because both sql queries directly return the input tuple.
%
But for queries that contain projection, this condition needs to be verified.

%how to do it.
The key here is that program $p_0$ and $p_1$ are loop-free and guaranteed terminated program due to the nature of
SQL queries.
%
Verifying relational properties for loop-free and guaranteed terminated programs is 
a well-studied problem in program language community.
%
Using satisfiability modulo theories(SMT) solver is a widely accepted technique for loop-free and guaranteed
terminated programs' verification problems.
%

%Explain SMT
Satisfiability modulo theories solver is a tool that decides if a given first order logic formula has an
interpretation for the variables that satisfy the formula.
%
If the formula is satisfiable, then the solver returns a model of variables that satisfy this formula.
%
For example, given the formula $x>0 /\ x < 5$, the SMT solver returns $SAT$, which means this formula is 
\textbf{satisfiable}.
%
Because $x$ can be $1$, and $1$ is greater then 0 and less than 5.
%
Given the formula $x > 10 /\ x < 5$, then SMT solver returns $UNSAT$, which means this formula is 
\textbf{unsatisfiable}.
%
Because there is no value for $x$ that can satisfy this formula.
%
For verifying the program properties, programs and properties are encoded as logic constrains, and using SMT
solver to check the satisfiability of the logic constrains to decides if the properties holds for the programs.

%
In this example, because the input is a vector of three elements, we create three variables $SEMPNO$,$SENAME$ and 
$SDEPTNO$ in the logic constrains that each 
corresponding an element in the input vector.
%
The condition of program $p_0$ reaches the return statement can be encoded into an first order logic constrains \\
$c_0 := SDEPTNO = 10 \land SDEPTNO + 5 > SEMPNO$.
%
And the logic constrains for program $p_1$ reaches the return statement is $c_1 := SDEPTNO = 10 \land 15 > SEMPNO$.
%
For verifying $p_1$ reaches the return statement implies $p_0$ reaches the return statement, 
we need to verify for any input that satisfying $c_1$ would satisfy $c_0$.
%
In other words, we need to prove that there is no input that satisfies $c_1$ and does not satisfy $c_0$.
%
So we feed the logical constrains $c_1 \land \neg c_0$ in the SMT solver. 
%
The SMT solver returns \textbf{unsat}, which means this logical constrains is unsatisfiable.
%
It indicates there is no such input that satisfies $c_1$ and does not satisfy $c_0$.

%
In the second step, we need to prove for the same input, if both program $p_0$ and $p_1$ reach the return statement,
then they return the same vector.
%
In the other words, we need to prove that there is no input, such that $p_0$ and $p_1$ return different vectors.
%
Because $p_0$ and $p_1$ both directly return the input vector, this statement holds trivially.
%
We can also encode the equivalent of the output as a logic constrains $Equal := SEMPNO = SEMPNO \land SENAME = 
SENAME \land  SDEPTNO \land SDEPTNO$.
%
Then we feed the constrains $\neg Equal \land c_1 \land c_0$ , which represent both programs reach the return 
statement and return different vector, 
into the SMT solver.
%
The SMT solver returns \textbf{unsat}, which indicates there is no such input for which the two programs return two 
different vectors.
%
Combing the proof that if $p_1$ reaches the return statement, then $p_0$ reaches the return statement.
%
We prove SQL query $q_0$ contains $q_1$.
%
By proving $q_1$ contains $q_0$ by the same approach, we prove $q_0$ is semantically equivalent to $q_1$.



\AJ{Problem formalization should come after overview.}
\section{Formalize Problem}\label{sec:formalize}
%
In this section, we give a formal definition of the semantic equivalence of SQL queries.

%
We first define the syntax of the target SQL query as the following:
%
\begin{align*}
  \sql ::=~&\STable{N}\\
         |~&\SFilter{P}{\sql}\\
         |~&\SProj{F}{\sql}\\
         |~&\SJoin{JoinType}{P}{\sql}{\sql}\\
         |~&\SUnion{\sql}{\sql} \\
         |~&\SAggregate{Agg}{\vec{Group}}{\sql}\\
\end{align*}

%
That is an sql query $Q$ can be a table operation, a filter operation, a project operation, a join operation, an union operation or an 
aggregate operation.
%
We choose this definition of SQL syntax is based on
\textbf{(1)} majority SQL can be converted into this form, and
\textbf{(2)} \sys is implemented on widely used open source framework calcite has similar internal representation for SQL query.
%

%
In order to define the semantic of these SQL queries, we need to define \textbf{relation}, which is its output and its input.
%
An relation $R$ is a bag of (multi-value set) of vectors.
%
Multi-value set means it allows the same vector appears multiple times.
%
An input $T$ for an SQL query $q_0$ is a set of tables and each table contains finite number of tuples.
%
An input $T$ is an valid input for $q_0$, if and only if it contains all tables that $q_0$ used.

We define the semantic of an SQL query in terms of its output relation on the valid input $T$.
\begin{gather*}
  \inference[$\textsc{E-Table}$]{%
  }{%
  \evaluation{T}{\STable{N}}{( t | \forall t \in N)}
  }
\end{gather*}
%
A table operation returns a relation that each vector corresponding a tuple in the table $N$.
%
\begin{gather*}
  \inference[$\textsc{E-Filter}$]{%
  \evaluation{T}{q_0}{R}
  }{%
    \evaluation{T}{\SFilter{P}{q_0}}{(v| \forall \vec{v} \in R,P(\vec{v}))}
  }
\end{gather*}
%
Given a filter operation $\SFilter{P}{q_0}$, if executing SQL query $q_0$ on valid input $T$ returns an relation $R_0$, the filter operation
returns a relations contains all vectors in $R_0$ that satisfy the predicate $P$.
%
\begin{gather*}
  \inference[$\textsc{E-Projection}$]{%
    \evaluation{T}{q_0}{R}
  }{%
    \evaluation{T}{\SProj{F}{q_0}}{( F(\vec{v}) | \forall \vec{v} \in R)}
  }
\end{gather*}
%
Given a project operation $\SProj{F}{q_0}$, if $q_0$ returns relation $R_0$ on $T$, it returns a relation by applying transformation function 
$F$ on all vectors in $R_0$.
%
An transformation function is the function that takes an vector as input, and outputs a vector.
%
In the system we support, the transformation function can apply variety operator on the elements of vectors, 
including arithmetic operation, \textbf{CASE} operator, inserting constant value includes \textbf{NULL}, deleting elements on certain indexes, adding 
element on certain indexes ,and user define functions. 
%
\begin{gather*}
  \inference[$\textsc{E-InnerJoin}$]{%
    \evaluation{T}{q_0}{R_0}\\
    \evaluation{T}{q_1}{R_1}\\
  }{%
    \evaluation{T}{\SJoin{Inner}{P}{q_0}{q_1}}{R_3}
  }
  \\
  R_3 :=(concat(\vec{v_0},\vec{v_1}) | \forall \vec{v_0} \in R_0, \forall \vec{v_1} \in R_1, P(\vec{v_0},\vec{v_1}))
  \\
  \inference[$\textsc{E-LeftJoin}$]{
   \evaluation{T}{q_0}{R_0}\\
   \evaluation{T}{q_1}{R_1}\\
  }
  {
   \evaluation{T}{\SJoin{Left}{P}{q_0}{q_1}}{R_3 \cup R_4}
  }
  \\
  R_4 :=(concat(\vec{v_0},\vec{null}) | \forall \vec{v_0} \in R_0 s.t. \forall \vec{v_1} \in R_1, \neg P(\vec{v_0},\vec{v_1}))
  \\
 \inference[$\textsc{E-FullJoin}$]{
   \evaluation{T}{q_0}{R_0}\\
   \evaluation{T}{q_1}{R_1}\\
  }
  {
   \evaluation{T}{\SJoin{Full}{P}{q_0}{q_1}}{R_3 \cup R_4 \cup R_5}
  }
  \\
  R_5 :=(concat(\vec{null},\vec{v_1}) | \forall \vec{v_1} \in R_1 s.t \forall \vec{v_0} \in R_0, P(\vec{v_0},\vec{v_1}))
\end{gather*}
%
Given a join operation $\SJoin{JoinType}{P}{q_0}{q_1}$ that $q_0$ returns $R_0$ and $q_1$ returns $R_1$, the semantic definition is based on the $JoinType$.
%
If it is a $Inner$ join, then it returns a relation $R_3$.
%
For all vector $\vec{v_0}$ in $R_0$, and all vector $\vec{v_1}$ in $R_1$, if $\vec{v_0}$ and
$\vec{v_1}$ are satisfied the predicate $P$, then the concatenation of $\vec{v_0}$ and $\vec{v_1}$ is in relation $R_3$.
%
If it is a $Left$ join, then it returns a relation which is union of $R_3$ and $R_4$.
%
$R_3$ has the same definition as the above.
%
$R_4$ contains all vector $\vec{v_0}$ in $R_0$, which cannot find a vector $\vec{v_1}$ in $R_1$ that satisfy predicate $P$, concat with a vector that has 
the length of $\vec{v_1}$ and each element is $null$.
%
For full outer join, it returns a union of $R_3$, $R_4$ and $R_5$.
%
$R_3$ and $R_4$ has the same definition as the above.
%
$R_5$ is defined symmetrically in terms of $R_4$.
%
For any type of join operation, the predicate $P$ is always a conjunction of equalities, that each equality has the form that left side is an index in 
$\vec{v_0}$ and right side is an index in $\vec{v_1}$.
%
\begin{gather*} 
  \inference[$\textsc{E-Union}$]{%
   \evaluation{T}{q_0}{R_0}\\
   \evaluation{T}{q_1}{R_1}\\
  }{%
  \evaluation{T}{\SUnion{q_0}{q_1}}{R_0 \cup R_1}
  }
\end{gather*}
%
Given a union operation $\SUnion{q_0}{q_1}$ that $q_0$ returns $R_0$ and $q_1$ returns $R_1$, it returns a relations that contains all vectors in $R_0$ and 
$R_1$.
%
Union operation requires $R_0$ and $R_1$ contains same length of vectors, and each corresponding element has the same type.
%
Union operation does \textbf{not} delete any duplicate vectors.
%
That is saying that union operation is always has the \textbf{ALL} keyword.
\begin{gather*}
  \inference[$\textsc{E-Aggregate}$]{%
   \evaluation{T}{q_0}{R_0}\\
  }{%
  \evaluation{T}{\SAggregate{Agg}{\vec{group}}{q_0}}{R_3}
  }
  \\
  R_3 := (Agg(\vec{g}) | \forall g \in partition(R_0,\vec{group}))
\end{gather*}
%
Given an aggregate operation $\SAggregate{Agg}{\vec{group}}{q_0}$, that $q_0$ returns $R_0$ on $T$, it returns a relation by applying 
aggregate function $Agg$ on each elements of the partition set $partition(R_0,\vec{group})$.
%
An aggregate function takes a set of vectors as the input, and output a vector.
%
In the system, we support common used aggregator in the aggregate function such as $MAX$, $MIN$, $COUNT$, $SUM$.
%
We also support use self-defined aggregator.
%
The element of the partition set $partition(R_0,\vec{group})$ is a set of vectors, that each set contains all vectors in $R_0$ that has the same 
elements of all indexes in $\vec{group}$.


%
In order to define \textbf{equivalent}, we need to first define \textbf{containment}.
%
\begin{mydef}
Given two SQL query $q_0$ and $q_1$, $q_0$ \textbf{contains} $q_1$, denote as $q_1 \subseteq q_0$, if and only if, 
for all all valid inputs $T$ for $q_0$ and $q_1$, 
such that relation $R_0$ is the executing result of $q_0$, and relation $R_1$ is the executing result of $q_1$ that respect $T$, then 
for all vectors $\vec{v}$ in $R_1$, $\vec{v}$ shows in $R_0$ at least one time. 
\end{mydef}
%
The definition of containment treats an relation as a set rather than a bag.
%
It means that for an vectors shows three times in the relation $R_1$, and only one times in the relation $R_0$, $R_0$ still contains $R_1$ based on the
definition.
% should we explain why it is good definition here?
\begin{mydef}
Given two SQL query $q_0$ and $q_1$, $q_0$ is \textbf{equivalent} to $q_1$, denote as $q_0 \equiv q_1$, if and only if 
$q_0 \subseteq q_1$ and $q_1 \subseteq q_0$.
\end{mydef}
Two queries are semantic equivalent if and only if they contains each other.
%
Because semantic equivalent use the definition of containment, semantic equivalent also treats an relation as a bag rather than a set.
%
In the rest of paper, we explain how to automatically verify two queries are equivalent by verifying two times containments relations under treating 
relations as sets


% bibliography:
\newcommand{\newblock}{}
\bibliographystyle{abbrv}
\small

\bibliography{p}


\end{document}
